
\chapter{Programming}

\pet{Much work needs to be done on this chapter.}

\CLAC\ is fully programmable. The language we designed for \CLAC\ was inspired by the Forth
computer language and the language used by the Hewlett-Packard HP-48SX calculator. If you are
used to a language such as C or Python, you will find \CLAC\ programs to be very strange
looking. However, the syntax used by \CLAC\ is a natural consequence of its stack oriented
nature. If you think in terms of the stack, programming \CLAC\ becomes very straightforward.

\CLAC\ can take its input from three sources.

\begin{description}
  
\item[Interactively.] The command line is \CLAC's source of interactive words. You can use all
  of \CLAC's programmable features on the command line. This is very useful for writing short
  loops.
  
\item[From .cpg files.] You can direct \CLAC\ to start reading words from a program file
  (default extension = .cpg) with the ``run'' command. For example

\begin{verbatim}
          => "myprog" run
\end{verbatim}
  
  will execute the program in the file ``myprog.cpg'' Notice that the quotes around ``myprog''
  are not really necessary due to the way \CLAC\ defaults all words to type String that it does
  not recognize.
  
  The text in a \CLAC\ program file can contain any sequence of words understood by \CLAC.
  
\item[From strings.] You can direct \CLAC\ to read words out of a string with the ``eval"
  command. For example

\begin{verbatim}
          => "pi *" eval
\end{verbatim}

     will execute the program contained in the string ``pi *''

\end{description}

These features, coupled with the basic control structures, and \CLAC's associative directory,
make \CLAC\ into a powerful and complete programming language.

\section{Control Structures}

Suppose you wanted to calculate the sum of the first 100 integers. You would use a loop, of
course.

\begin{verbatim}
     => 0 1 100 for I + next
\end{verbatim}

The first zero is the accumulator. The next two numbers represent the loop's range. In this
case, the range is from 1 to 100. The ``for'' word starts the loop. It pops the range values
from the stack thus leaving the zero in stack level 1.

Inside the loop, the word ``I'' means the current loop counter. The first time through the loop
I will be 1. The ``next'' word advances the value of I and loops back to the word just after
``for.'' If it's time for the loop to exit, ``next'' does not loop back, of course, but just
continues on.

The ``next'' does not have to be on the same line as the ``for.'' In fact, you could enter as
many lines as you wanted between a for and its next. You can do arbitrarily complicated
calculations between a for and its next, including other loops.

Suppose you wanted to calculate

\begin{displaymath}
     \frac{1}{2} + \frac{2}{3} + \frac{3}{4} + \frac{4}{5} + \cdots + \frac{99}{100}
\end{displaymath}

No problem.

\begin{verbatim}
     => fraction 0 1 99 for I I 1 + / + next
\end{verbatim}

It looks strange. But it works.

Suppose you wanted to calculate the above series often, but with different bounds. One way is to
put the meaty part of the program into a string.

\begin{verbatim}
     => "for I I + / + next" Series sto
\end{verbatim}

Now you can evaluate for various bounds like this

\begin{verbatim}
     => 0 1 99 Series eval
\end{verbatim}

When \CLAC\ sees the word ``Series'' it calls up the string you defined from the directory. The
``eval'' word causes the text of that string to be interpreted as a program. Since the first
word in that program is ``for'', the 1 and 99 on stack levels one and two are taken to be the
limits of the for loop. Notice that it's still necessary to put an initial zero on the stack
outside of the evaluated string. All the eval effectively does is expand the string so that the
sequence of words \CLAC\ sees includes the words in that string.

You can thus use strings for storing quick and dirty programs.
