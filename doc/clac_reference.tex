% ===========================================================================
% FILE         : clac_reference.tex
% SUBJECT      : Reference Manual for CLAC
% LAST REVISED : June 2002
% AUTHOR       : Peter Chapin
%
% This is the reference manual for the CLAC program. It is intended to
% provide a complete description of the program.
% ===========================================================================

% ++++++++++++++++++++++++++++++++
% Preamble and global declarations
% ++++++++++++++++++++++++++++++++
\documentclass{report}
\pagestyle{headings}
\setlength{\parindent}{0em}
\setlength{\parskip}{1.75ex plus0.5ex minus0.5ex}

% ===========================================================================
% FILE   : macros.tex
% SUBJECT: This file contains useful macros used in all the Clac documentation.
% AUTHOR : Peter C. Chapin
%
% ===========================================================================

\newcommand{\CLAC}{Clac}               % Formatting the name of the program.
\newcommand{\newterm}[1]{\emph{#1}}    % For newly introduced terms.
\newcommand{\code}[1]{\texttt{#1}}     % Used for bits of inline code.
\newcommand{\filename}[1]{\texttt{#1}} % For file names.
\newcommand{\todo}[1]{\textit{#1}}     % For quick, general to do items.

% The \note macros are useful for creating easy to see notes.

% Peter Chapin
\long\def\pet#1{\marginpar{PET}{\small \ \ $\langle\langle\langle$\
{#1}\
    $\rangle\rangle\rangle$\ \ }} 

%% The following are settings for the listings package. See the listings package documentation
%% for more information.
%%
%\lstset{language=C,
%        basicstyle=\small,
%        stringstyle=\ttfamily,
%        commentstyle=\ttfamily,
%        xleftmargin=0.25in,
%        showstringspaces=false}


% +++++++++++++++++++
% The document itself
% +++++++++++++++++++
\begin{document}

% ----------------------
% Title page information
% ----------------------
\title{\CLAC\ Reference Manual}
\author{Peter Chapin\thanks{PChapin@vtc.vsc.edu}\\
        Peter Nikolaidis\thanks{peter@paradigmcc.com}}
\date{June 22, 2002}
\maketitle

% ------------------------
% Make a table of contents
% ------------------------
\pagenumbering{roman}
\tableofcontents
\newpage
\pagenumbering{arabic}

% -------------
% Chapter Break
% -------------
\chapter{Introduction}

\section{Overview}

\CLAC\ is a general purpose scientific and engineering calculator program for Unix and Windows
systems. \CLAC\ was inspired by the Hewlett Packard line of calculators. The HP- 48SX was
particularly influential. However, \CLAC\ is not a true simulation of any existing hand held
calculator. Not all of the abilities of, for example the HP-48SX, have been implemented.
Furthermore, \CLAC\ offers features not found on any current hand held unit.

Some of \CLAC's nicer features include:

\begin{enumerate}
  
\item Operations can be performed on a variety of data types. Where it is meaningful to do so,
  operations can easily be performed on mixed data types.
  
\item A tree structured directory is maintained in memory where objects of all types may be
  saved. The entire tree can be written to disk for later retrieval.

\item Individual objects can be saved on disk.
  
\item Programs can be written in a FORTH-like language to extend the functionality of \CLAC.

\end{enumerate}

This document is a complete user reference manual for \CLAC. We assume that you are familiar
with \CLAC's basic operation. In particular, we assume that you are comfortable with the stack
oriented nature of \CLAC. In addition, we assume that you are familiar with operating system
concepts such as pathnames, script files, and I/O redirection. If you are not familiar with
\CLAC's basic operation, you should work through the tutorial document first.

This document also assumes that you are familiar with the appropriate mathematical material.
This document is not a tutorial, for example, on matrix algebra or complex numbers.

\section{Legal Note}

%
% Put the GPL here?
%

% -------------
% Chapter Break
% -------------
\chapter{Entity Library}

\CLAC\ manipulates several different types of objects. The effects of an operation varies from
type to type. For example, the effect of multiplying two float objects is rather different than
that of multiplying two matrix objects. Nevertheless, objects of differing types can coexist on
\CLAC's primary stack, in \CLAC's directory, and in .CLC files.

A complete list of all of \CLAC's object types, along with some of their basic properties
follows.

\section{Integer}

Integer's are whole numbers which may be positive or negative. They are always displayed without
a decimal point. \CLAC\ manages integers with arbitrary precision.

\section{Rational}

Rational numbers are fractions which may be positive or negative. They are always displayed as
two integers separated by a `/' character. For example:

\begin{verbatim}
          -4/5
\end{verbatim}

\CLAC\ does not transform fractions into floating point numbers when simple operations are
performed entirely with fractions. For example if $2/3 - 1/6$ is calculated, the result will be
$1/2$, not $0.500$. \CLAC\ always displays rationals in reduced form.

\CLAC\ manages the numerator and denominator of rational objects with arbitrary precision. It is
possible for a rational to have, for example, a 300 digit numerator and a 40,000 digit
denominator.

\section{Float}

Float numbers are signed numbers which contain a mantissa portion and an exponent portion.
Although \CLAC\ offers several display modes for float numbers, a decimal point is always
involved in the display. The range of float numbers is $-1.7\times 10^{307} \cdots -1.7\times
10^{-308}, 0.0, +1.7\times 10^{-308} \cdots +1.7\times 10^{+307}$.

Notice that a float number can take on the ``exact'' value zero as a special case.

\CLAC\ manages float numbers with 16 decimal digits of precision.

\section{Complex}

Complex numbers are composed of two floating components. One of the components is the real part
and the other component is the imaginary part. Complex numbers are always displayed inside of
parenthesis. For example $(3.121, 6.774)$ means $3.121 + i 6.774$.

\section{Binary}

Binary numbers are specialized unsigned integers. \CLAC\ performs operations on binary numbers
as if they contained exactly the number of bits you specify. Binary numbers may be displayed in
several bases. However, a leading `\#' character is always used to identify the number as a
binary. For example, assuming that you have selected a word size of 16 bits, the following
binary numbers are identical.

\begin{tabbing}

\hspace*{3em}\=\#FFFEh\hspace{6em}\=Trailing `h' denotes hexidecimal base.\\
\>             \#65534d\>           Trailing `d' denotes decimal base.\\
\>             \#177776o\>          Trailing `o' denotes octal base.\\
\>             \#1111111111111110b\>Trailing `b' denotes binary base.

\end{tabbing}

If the calculation \#FFFEh + \#0003h is done, the result will be, for 16 bit word size, \#0001h.
Notice that the final carry is ignored.

\section{String}

Strings are sequences of characters enclosed in quotation marks. Any character except the
quotation mark is allowed in a string. In most cases, \CLAC\ does not try to interpret the text
of a string. Strings may be arbitrarly long.

Strings are used by \CLAC\ as variable names, filenames, and operating system commands. You may
find them useful for storing arbitrary notes as well.

\section{List}

Lists are sequences of other calculator objects enclosed in $\{\ldots\}$ pairs. The objects in
the list are delimited by spaces. The objects in a list may be of any type---including other
lists. For example, the following is a valid list:

\begin{verbatim}
          { 1 -7/8 2.33 (4.33, 6.55) # FFh "Hi" { 1 "Sublist" } }
\end{verbatim}

Lists are used by \CLAC\ primary as a means of grouping objects into a single object. This is
useful when you want to store several objects into one directory entry, save several objects in
one .CLC file, or pass several objects into a subprogram.

Lists have no limitation or their size or on the size or complexity of their elements.

Notice that lists are ordered.

\section{Matrix}

Matrices are two dimensional arrays of other calculator objects. Each matrix is enclosed in
$[\ldots]$ pairs and, if there is more than one row in the matrix, each row is also enclosed in
$[\ldots]$ pairs. The objects in a matrix are delimited by spaces. The elements of a matrix may
be of any type except other matrices. For example, the following are valid matrices:

\begin{verbatim}
          [ 1 -2/3 3.142 ]         A 1x3 matrix.
          [ [ 1 ] [ 2 ] [ 3 ] ]    A 3x1 matrix.
          [ [ 1 2 3 ] [ 4 5 6 ] ]  A 2x3 matrix.
\end{verbatim}

The restriction on the types of the matrix elements is necessary to prevent ambiguity. For
example, if matrices where allowed as matrix elements, the last example above could also be
interpreted as a 1x2 matrix of 1x3 matrices.

Matrices differ from lists in two important respects. First, matrices are two dimensional.
Second, most operations have very different effects when applied to matrices or to lists. For
example, adding two matrices is only allowed if their dimensions are compatible. In that case,
the sum matrix is calculated by adding the corresponding matrix elements from the operands. In
contrast, adding two lists simply forms a new list in which the elements of the first list are
followed by the elements of the second list.

As with lists, \CLAC\ has no arbitrary limits on the dimensions of a matrix or on the size or
complexity of the individual matrix elements. Notice that matrices of lists are possible (as are
lists of matrices as well as matrices of lists of matrices, etc).

\CLAC\ uses matrices with a row (or column) dimension of one for vectors. \CLAC\ does not
support a special vector type. Notice that transforming a row vector into a column vector can be
done with the ``trans'' (transpose) command.


% -------------
% Chapter Break
% -------------
\chapter{Type Conversions}

% -------------
% Chapter Break
% -------------
\chapter{Input Model}

% -------------
% Chapter Break
% -------------
\chapter{Modes and Flags}

% -------------
% Chapter Break
% -------------
\chapter{Configuration File}

% -------------
% Chapter Break
% -------------
\chapter{Unary Operations}

\CLAC\ supports the following unary operators. All unary operators take their operand from stack
level 1 and return their result to stack level 1.

\begin{tabbing}
\hspace*{3em}\=abs\hspace{3em}\=Absolute value\\
\>             acos\>           Arc cosine\\
\>             asin\>           Arc sine\\
\>             atan\>           Arc tangent\\
\>             conj\>           Complex Conjugate\\
\>             cos\>            Cosine\\
\>             exp\>            $e^{x}$\\
\>             exp10\>          $10^{x}$\\
\>             fp\>             Fractional part\\
\>             im\>             Imaginary part\\
\>             ip\>             Integer part\\
\>             inv\>            Inverse\\
\>             ln\>             Natural logarithm\\
\>             log\>            Logarithm\\
\>             neg\>            Negate\\
\>             NOT\>            Logical NOT\\
\>             re\>             Real part\\
\>             sgn\>            Sign\\
\>             sin\>            Sine\\
\>             sq\>             $x^{2}$\\
\>             sqrt\>           $\sqrt{x}$\\
\>             tan\>            Tangent\\
\end{tabbing}

% -------------
% Chapter Break
% -------------
\chapter{Binary Operations}

\CLAC\ supports the following binary operators. All binary operators take one operand, ``x,''
from stack level 1 and another operand, ``y,'' from stack level 2. The operation returns its
result to stack level 1.

\begin{tabbing}
\hspace*{3em}\=+\hspace{3em}\=$y + x$\\
\>             -\>            $y - x$\\
\>             *\>            $y   x$\\
\>             /\>            $y / x$\\
\>             ?\>            $y^{x}$\\
\>             mod\>          $y$ modulo $x$\\
\>             AND\>          Logical AND\\
\>             OR\>           Logical OR\\
\>             XOR\>          Logical XOR\\
\end{tabbing}

% -------------
% Chapter Break
% -------------
\chapter{Action Operations}

\begin{tabbing}
\hspace*{3em}\=bin\hspace{3em}\=Display binary objects in binary\\
\>             clear\>          Clear the stack of all elements\\
\>             dec\>            Display binary objects in decimal\\
\>             deg\>            Sets degrees angle mode\\
\>             drop\>           Deletes level stack level 1\\
\>             dropn\>          Deletes n levels of the stack\\
\>             dup\>            Duplicate stack level 1\\
\>             dupn\>           Duplicate n levels of the stack\\
\>             fix\>            Display floating point numbers with n decimal places\\
\>             eng\>            Display floating point numbers in engineering notatio\\
\>             grad\>           Sets gradians angle mode\\
\>             hex\>            Display binary objects in hexadecimal\\
\>             oct\>            Display binary objects in octal\\
\>             polar\>          Sets polar display mode for complex objects\\
\>             rad\>            Sets radians angle mode\\
\>             read\>           Uses a string in stack level 1 as a file to read from\\
\>             rec\>            Sets rectangular display mode for complex objects\\
\>             rolld\>          Moves the level 2 object to level n\\
\>             roll\>           Moves the level n+1 object to level n\\
\>             rot\>            Moves the level 3 object to level 1\\
\>             sci\>            Display floating point numbers in scientific notation\\
\>             stws\>           Sets the binary integer wordsize\\
\>             swap\>           Swaps the objects in levels one and two\\
\>             sys\>            Uses a string in stack level 1 as a DOS command\\
\>             write\>          Uses a string in stack level 1 as a file to write to\\
\>             sl\>             Logical shift left\\
\>             sr\>             Logical shift right\\
\end{tabbing}

% -------------
% Chapter Break
% -------------
\chapter{File and Directory Operations}

% -------------
% Chapter Break
% -------------
\chapter{Programming}

% -------------
% Chapter Break
% -------------
\chapter{Conclusion}

% -----------------------------------
% Here is where the appendicies start
% -----------------------------------
\appendix

% -------------
% Chapter Break
% -------------
\chapter{Error Messages}

% -------------
% Chapter Break
% -------------
\chapter{Format of .CLC Files}

% -------------
% Chapter Break
% -------------
\chapter{Future Directions}

% -------------
% Chapter Break
% -------------
\chapter{\CLAC\ Quick Reference}

\end{document}

