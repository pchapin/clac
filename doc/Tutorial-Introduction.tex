
\chapter{Introduction}

\CLAC\ is a general purpose scientific calculator program. \CLAC\ also has features that would
be of interest to programmers and computer scientists. \CLAC\ offers the following significant
features:

\begin{enumerate}
  
\item \CLAC\ has a stack oriented design that makes it easy to evaluate complex expressions.
  
\item \CLAC\ supports the usual scientific functions.
  
\item \CLAC\ allows you to use several different data types in your calculations. For example,
  \CLAC\ allows you to manipulate complex numbers, matricies, fractions, and huge integers as
  easily as floating point numbers.
  
\item \CLAC\ offers a tree structed directory for storing the results of calculations.
  
\item \CLAC\ is fully dynamic. It has few arbitrary limits. \CLAC\ is only limited by the size
  of available memory.
  
\item \CLAC\ is fully programmable. Using a language very similar to Forth, \CLAC\ supports
  conditional branching, loops, subprograms, arrays, pointers, structures, and more.

\end{enumerate}

We intend for \CLAC\ to be used primarily interactively. Although highly programmable, \CLAC\
programs tend to be slow and hard to read. Furthermore some operations are awkward to program.
We expect that you will use \CLAC\ programs to streamline your interaction with \CLAC. If you
want to do serious numerical work, you might use \CLAC\ as a prototyping tool prior to writing a
program in another language.

\CLAC\ does not support certain features commonly found on today's high end calculators. In
particular, \CLAC\ does not do symbolic manipulation or graphing. We may implement these
features in future versions of \CLAC.

In addition, \CLAC\ does not directly support a number of functions that are built into many
calculators. For example, \CLAC\ does not offer any statistical or financial functions. You can
implement many of these functions yourself using \CLAC's programming language. This tutorial
demonstrates several possibilities.

This document is a complete tutorial on using \CLAC. First, we discuss the basics of calculating
on a stack machine. Next we discuss the data types and built-in operations that \CLAC\ supports
and what you can do with them. Finally, we discuss how to program \CLAC.

Remember that this document is only a tutorial. We do not describe \CLAC\ completely in this
document. See the reference manual for more detailed information.
