% ===========================================================================
% FILE         : Design.tex
% SUBJECT      : Document that describes the design of CLAC.
% LAST REVISED : 2022-12-13
% AUTHOR       : (C) Copyright 2022 by Peter C. Chapin and Peter Nikolaidis
%
% This is the design description for the CLAC program.
% ===========================================================================

% ++++++++++++++++++++++++++++++++
% Preamble and global declarations
% ++++++++++++++++++++++++++++++++
\documentclass{report}
\pagestyle{headings}
\setlength{\parindent}{0em}
\setlength{\parskip}{1.75ex plus0.5ex minus0.5ex}

% ===========================================================================
% FILE   : macros.tex
% SUBJECT: This file contains useful macros used in all the Clac documentation.
% AUTHOR : Peter C. Chapin
%
% ===========================================================================

\newcommand{\CLAC}{Clac}               % Formatting the name of the program.
\newcommand{\newterm}[1]{\emph{#1}}    % For newly introduced terms.
\newcommand{\code}[1]{\texttt{#1}}     % Used for bits of inline code.
\newcommand{\filename}[1]{\texttt{#1}} % For file names.
\newcommand{\todo}[1]{\textit{#1}}     % For quick, general to do items.

% The \note macros are useful for creating easy to see notes.

% Peter Chapin
\long\def\pet#1{\marginpar{PET}{\small \ \ $\langle\langle\langle$\
{#1}\
    $\rangle\rangle\rangle$\ \ }} 

%% The following are settings for the listings package. See the listings package documentation
%% for more information.
%%
%\lstset{language=C,
%        basicstyle=\small,
%        stringstyle=\ttfamily,
%        commentstyle=\ttfamily,
%        xleftmargin=0.25in,
%        showstringspaces=false}


% +++++++++++++++++++
% The document itself
% +++++++++++++++++++
\begin{document}

% ----------------------
% Title page information
% ----------------------
\title{The Design of CLAC}
\author{Peter Chapin\thanks{chapinp@proton.me}\\
        Peter Nikolaidis\thanks{peter@paradigmcc.com}}
\date{December 1, 2007}
\maketitle

% ------------------------
% Make a table of contents
% ------------------------
\pagenumbering{roman}
\tableofcontents
\newpage
\pagenumbering{arabic}

% -------------
% Chapter Break
% -------------
\chapter{Introduction}

This document describes the design of the \CLAC\ calculator program. In this document we cover
both the general features of the design as well as its details. This document is intended for
programmers who are interested in understanding \CLAC's internals or who are considering
modifying \CLAC\ in some way. We assume you are familiar with C++ programming. This document is
not a tutorial on that language. It would also be useful if you were familiar with the usage of
\CLAC.

\CLAC\ was conceived by Peter Chapin and Peter Nikolaidis in the early 1990s. The idea was to
build a program that provided the same functionality as a Hewlett-Packard HP-48 calculator.
However, \CLAC\ was not intended to simulate the HP-48, run HP-48 programs, or present a user
interface that was identical to that of the HP-48. Since \CLAC\ was to be a program for a
general purpose computer, the hope was that it could take advantage of the richer environment
such a machine offered and have features that would not be easily implemented on a hand held
calculator. The HP-48 was the inspiration for \CLAC, but it never defined the requirements for
\CLAC.

As \CLAC\ developed, a number of HP-48 features were abandoned, at least temporarily, in the
interests of keeping \CLAC\ manageable. However, \CLAC\ also evolved toward greater flexibility
and integration with other software than any kind of ``mere'' calculator simulation could have
provided. While many interesting features of the HP-48 are still missing from its design, we
believe \CLAC\ more than makes up for that in other ways.

\CLAC\ is divided into three major parts. The first part is a library of ``entities'' that
represent the various kinds of objects \CLAC\ can operate on. This library is not dependent on
the rest of \CLAC. It is possible to use the \CLAC\ entity library in another program of your
own design. While \CLAC\ entities are currently optimized for use in an interactive environment
where performance concerns are secondary, they could be used as part of a general purpose
numerical library.

The second part of \CLAC\ is the execution engine that reads a stream of ``words'' and performs
the operations indicated by that stream. The execution engine implements \CLAC's scripting
language while using the entity library to perform the actual computations. By adding a very
simple user interface to the execution engine, \CLAC\ could be made into a specialized numeric
scripting language.

The third part of \CLAC\ is the high level user interface (UI). The current UI is a simple,
line-oriented interface that accepts a line of \CLAC\ command words and outputs information
about the top of \CLAC's stack. More complex user interfaces with graphical front ends could
easily be imagined and potentially added to \CLAC\ later.

The entity library and the execution engine are both highly portable. As much as possible they
only use features of Standard C++. In contrast the user interface is highly system dependent.
However, it should be relatively easy to modify \CLAC\ for a different user interfaces since the
user interface can be changed independently of the execution engine or the entity library. The
modular nature of the design makes it easy to adapt \CLAC\ to a variety of application domains.
This is one of \CLAC's strengths.

The rest of this manual describes the three components of \CLAC\ in more detail. The approach we
follow here is bottom-up. We start by talking about the entity library, followed by the
execution engine, and then we end by talking about the user interface. In some ways this is the
opposite of how the user perceives the program. However, we feel that from the programmer's
perspective this organization makes the most sense.

% -------------
% Chapter Break
% -------------
\chapter{Entity Library}

\section{Overview}

\CLAC\ is distinctive in that it allows you to manipulate objects of a wide variety of types in
a uniform way. Objects of various types may be mixed freely on the stack, in directories, or in
aggregate objects such as lists, vectors, or matricies. \CLAC\ manages this by defining an
abstract base class named Entity. Although no objects of type Entity exist in the program, The
stack, directories, and aggregate objects are all implemented as collections of Entity pointers.
Since each of the object types are derived from Entity, these collections may contain pointers
to objects of various actual types mixed freely.

Virtual functions declared in class Entity define the operations that can be applied to all the
objects. For example it is possible to call \texttt{abs()} for any Entity type to generate a new
object that is the absolute value of the original.

In addition, virtual functions declared in Entity support conversion to any of the other types.
For example it is possible to call \texttt{to\_float()} for any Entity type to generate a new
Float\_Entity object that has the same value as the original.

Finally, virtual functions declared in Entity support binary operations. The left operand is the
``this'' object while the right operand is given by the pointer to Entity argument to the
function. All the binary operations are designed to work on objects of the same type. Thus, even
though the argument to \texttt{plus()}, for example, is pointer to Entity, The \texttt{plus()}
in class Integer\_Entity can only receive a pointer to an Integer\_Entity as an actual
parameter. This is checked at run time using a dynamic cast. This design decision was made to
limit the number of mixed type binary operations that had to be implemented. There are currently
no such operations.

It is important to understand that every operation on an entity returns a new entity that
contains the resulting value. Operations never change an entity. In fact, entity objects are
immutable once they have been constructed. This was done to facilitate error handling. If a
result can't be computed for some reason, the operation will throw an exception. In that case,
the entity being operated on will, of course, remain unchanged and well defined.

Every entity operation returns a pointer to a new entity but the actual type of the entity
returned is not made explicit. This is because in some cases the returned entity has a different
type than the original entity. For example, taking the \texttt{sin()} of an Integer\_Entity will
return a Float\_Entity. However, multiplying two Integer\_Entities together will return another
Integer\_Entity. In some cases the exact type returned may be a function of the argument values
given to an operation. For example taking the \texttt{sqrt()} of a Float\_Entity will return a
new Float\_Entity if the argument value is zero or above, but a Complex\_Entity if the argument
value is negative.

It is also important to realize that all entities are allocated dynamically on the heap.
\emph{This rule must be followed!} Entity objects can always be safely cleaned up with the
\texttt{delete} operator and \CLAC\ assumes this throughout.

To make these comments more concrete, consider the abbreviated class definition below.

\begin{verbatim}
class Entity {
  public:
    virtual ~Entity() = 0;

     // Unary operations.
     virtual Entity *abs() const;

     // Conversion operations.
     virtual Entity *to_integer() const;
     virtual Entity *to_float() const;

     // Binary operations.
     virtual Entity *plus(const Entity *) const;
  };
\end{verbatim}

Not all of these operations are meaningful for all the types. For example, what does
\texttt{abs()} as applied to Directory\_Entity mean? Thus all the operation methods have a
default implementation in class Entity that returns a failure indication (more on this below).
If an operation is not meaningful for a type it is simply not implemented---it will inherit the
error producing implementation from class Entity. This design makes it feasible to include all
operations as virtual functions in class Entity. Even operations, such \texttt{det()}, which
only apply to one type (Matrix\_Entity) are handled in this way. This allows the higher levels
of the program to apply any operation to any object without worrying about the actual types
involved. The objects will return error information if the operation cannot be done. Ultimately
this error information will need to be handled by the user interface.

Another consequence of this design is that if a new type is introduced, only the operations that
can be meaningfully applied to it need be implemented. The higher levels of the program need not
know that a new type is present to be able to work with objects of that type. If the user tries
to perform a meaningless operation on a type, that error is detected at run time. As far as the
compiler is concerned all operations can be applied to any Entity.

In addition adding a new operation is easy. One must add a suitable virtual function in class
Entity and provide an exception throwing default implementation for it. All the existing Entity
classes will inherit this operation. For those classes where it makes sense to support the new
operation, an overriding function can be added.

The Entity methods are all const methods. Furthermore, the binary operation methods take a
pointer to a const as a parameter. This reflects the convention that the operation functions
return a fresh object as a result. The operand(s) of an operation are never changed by that
operation. If the operation cannot be completed, the method returns an error indication instead
of a pointer to any kind of result object.

\section{Entity Types}

What follows is a catalog of all the types in \CLAC\ that are derived from Entity. For each type
we have included some notes about the implementation of that type.

\subsection{Binary\_Entity}

This type is implemented as a simple long integer as defined by the C++ compiler used to compile
\CLAC. It is used for special ``binary'' operations.

\subsection{Complex\_Entity}

This type is implemented using the complex template in the C++ standard library. In particular
it uses std::complex$<$double$>$.

\subsection{Directory\_Entity}

This type is implemented as a standard map that maps std::string to a pointer to an entity.
\CLAC\ uses directories as storage areas. Since directories are entities themselves, nested
directories are fully supported.

\subsection{Float\_Entity}

This type is implemented using the C++ built in type double. Double is well supported in the
standard library and is satisfactory for many calculations. A future version of the Entity
library may use (or offer) an extended precision floating point type.

\subsection{Integer\_Entity}

This type is implemented using a single object of type very\_long. The very\_long class
implements arbitrary precision signed integers with memory use that grows and shrinks
dynamically as needed. Integer\_Entities watch their contained very\_long objects and report
errors as appropriate.

\subsection{Labeled\_Entity}

A Labeled\_Entity binds a label as a std::string to some other Entity. It can be used to tag an
Entity with some information about that Entities meaning or purpose. Operations on a
Labeled\_Entity are all forwarded to the contained Entity.

\subsection{List\_Entity}

This type is implemented using a std::list$<$Entity*$>$. Thus Lists of any type (including other
lists) are supported.

\subsection{Matrix\_Entity}

This type is implemented as a list of list of pointers to Entity. This general implementation
supports matricies of arbitrary and changing size as well as matricies containing arbitrary
types of elements. Since matricies must be rectangular, Matrix\_Entities perform the necessary
adjustments internally to insure that this is always so.

\subsection{Program\_Entity}

This type is implemented in a way similar to String\_Entity. Many of the operations are
different when applied to Program\_Entity, however. Program entities are used to store \CLAC\
program text.

\subsection{Rational\_Entity}

This type is implemented using a single object of type rational$<$very\_long$>$. This allows for
rational numbers with very large numerators or denominators. Note that the rational template
allows rational numbers to be created using any underlying integral type. However. \CLAC\ only
uses the template with very\_long.

\subsection{String\_Entity}

This type is implemented using an object of type std::string from the C++ standard library.

\subsection{Vector\_Entity}

This type is also implemented using a std::list$<$Entity*$>$. The operations are implemented
differently for vectors, however. Note that since Vector\_Entity is implemented as a dynamic
list, vectors of arbitrary dimension are supported. Since Vector\_Entity is implemented as a
list of Entity*, vectors with arbitrary (even mixed) element types are supported.

\section{Type Conversion}

When two objects meet during a binary operation, \CLAC\ will convert them both to a common type
before applying the binary operator function. This insures that the binary operator methods only
need to support operands of the same type. Note that a conversion may be applied to both
operands. Thus either the left, the right, or both operands may be promoted before the binary
operation begins.

\CLAC\ decides what conversions to apply by consulting a conversion matrix. It selects the row
of the matrix using the type of the left operand, and it selects the column of the matrix using
the type of the right operand. The matrix element specifies which conversion function is to be
applied to both operands. Only if both conversions are successful does \CLAC\ attempt to run the
binary operation.

Here is, roughly, how the conversion matrix looks

%
% Use a tabbing environment
%
\begin{verbatim}
               INTEGER        FLOAT
     INTEGER:  to_integer     to_float

     FLOAT  :  to_float       to_float
\end{verbatim}

This matrix shows that whenever float entities and integer entities meet, the conversion
\texttt{to\_float()} is to be applied to both. Thus integers are promoted to floats. The matrix
must be symmetric about the main diagonal. Note that the matrix is a two dimensional array of
pointers to virtual members of class Entity\footnote{This relatively exotic data type exposed
  some compiler bugs in an early compiler used during \CLAC's development.}.

We chose this design to solve an almost intractable problem. We did not want to implement a
binary operation function for every possible combination of types. For example, in class
Integer\_Entity we did not want to have a \texttt{plus()} for each of the possible right operand
types. One alternative would be something like

\begin{verbatim}
Entity *Integer_Entity::plus(const Entity *right) const
  {
    switch (right->my_type()) {
      case INTEGER:
        // Implement INTEGER+INTEGER.
        break;

      case FLOAT:
        // Implement INTEGER+FLOAT.
        break;

      // etc...
    }
  }
\end{verbatim}

However, this is extremely messy. It's no better than implementing a different version of
\texttt{plus()} for each right operand type. It's very redundant since in many cases there are
similar conversions, etc. It also greatly increases coupling between modules. For example the
translation unit implementing Integer\_Entity would have to include headers for all the other
types. Not only that, the very similar operations of INTEGER+FLOAT and FLOAT+INTEGER would be
implemented in different translation units! Adding a new type would require editing \emph{all}
the translation units used to implement the other types.

Normally when you see a large switch statement based on a type field, you should use virtual
functions and derived classes. This idea does not work here. The virtual function mechanism is
already being used to select the correct \texttt{plus()} version based on the actual type of the
left operand. Fundamentally, the virtual function mechanism is a one dimensional process. This
is a two dimensional problem. In fact, the conversion table is, essentially, trying to be a
manually constructed two dimensional vtable.

The solution we settled on uses run time type identification to select an appropriate
conversion. However, since the conversion is a virtual function, \CLAC\ need not remember the
results of the run time type query. There are no ugly switch statements in \CLAC\ of the sort
shown above. For example:

\begin{verbatim}
     // Look up conversion based on types.
     F = convert_table[left->my_type()][right->my_type()];

     // No longer care about the actual types!
     Entity *tempL = left ->*F();
     Entity *tempR = right->*F();
\end{verbatim}

Now adding a new type is fairly simple. The conversion table needs to be augmented and the
appropriate functions in the new type need to be implemented. The existing types need not be
touched (except possibly to add conversion functions to the new type). Furthermore, the
implementation of the new type need not include all the other headers.

The price paid is flexibility. \CLAC\ cannot actually do binary operations on mixed types. All
mixed calculations must be resolved into a calculation on objects of a common type. This is
sometimes not desirable. For example to support the multiplication of a matrix by a scalar, it
was necessary to write special case code in the matrix \texttt{multiply()} method. If either
operand is a 1x1 matrix, \CLAC\ assumes it is the result of a promoted scalar and does the
corresponding operation. As a result \CLAC\ will incorrectly multiply an actual 1x1 matrix, for
example as entered explicitly by the user, with a 3x3 matrix (rather than produce an error
message as it should). Fortunately quirks of this sort are relatively rare and we believe them
to be minor.

\section{Error Handling in Entities}

Error handling is, in general, difficult. \CLAC\ deals with errors in entity operations by using
just two simple rules.

\begin{enumerate}
  
\item Each entity operation returns a pointer to a new object (perhaps of different type) that
  contains the result of the operation. Even if the operation doesn't change the data in the
  object, the operation must still return a pointer to a \emph{new} object. This uniform
  treatment will be important to any code that manipulates Entity pointers generically.
  \emph{The original object} \emph{is unchanged}.
  
\item If the operation fails, it throws an exception. Thus if the operation returns, it worked.
  There is no need to test for NULL pointers being returned from entity operations.

\end{enumerate}

With these semantics, the main part of the program looks like this:

\begin{verbatim}
try {
  <Attempt an operation on the entity in stack level 0>
  <Delete stack level 0>
  <Install the pointer obtained above into stack level 0>
}
catch (Entity::Error) {
  error_message("Oops, I couldn't do that!");
  // Object in stack level 0 unchanged, just loop back and go on.
}
\end{verbatim}

Now let's take a look at a couple of details...

For operations that return the same entity they are invoked against we need to return a pointer
to a duplicate of *this. It so happens we have a duplication method for this purpose.

\begin{verbatim}
Entity *Binary_Entity::abs()
{
  return duplicate();
}
\end{verbatim}

This works as desired. If \texttt{duplicate()} fails, it will throw an exception which will pass
out of Binary\_Entity::abs() as desired. *this is not changed and there is nothing hanging
around that needs cleaning up.

Consider composite operations. Suppose we tried the following for Integer\_Entity::inv() :

\begin{verbatim}
Entity *Integer_Entity::inv()
{
  return to_float()->inv();
}
\end{verbatim}

This doesn't quite work. The \texttt{to\_float()} function returns a pointer to a \emph{new}
object. That pointer is immediately used and another new object is created to hold the inverse.
The pointer to the inverse is returned, but the object used to hold the Float\_Entity version of
the original Integer\_Entity is left hanging around. Thus the code above has a memory leak. Here
is the way to fix that:

\begin{verbatim}
Entity *Integer_Entity::inv()
{
  std::auto_ptr<Float_Entity> p = static_cast<Float_Entity *>(to_float());
  return p->inv();
}
\end{verbatim}

This solution insures that the temporary Float\_Entity is properly deleted even in the case
where an exception is thrown by either the \texttt{to\_float()} or \texttt{inv()} operations.


% -------------
% Chapter Break
% -------------
\chapter{The Execution Engine}

\section{Word Stream}

\CLAC's user interface accepts keystrokes and mouse events from the user and dispatches them to
the appropriate window via virtual function calls. However, the \CLAC\ execution engine wants to
process an unending stream of words.

Typically these words come from the command window. The command window maintains a std::string
(actually a std::list$<$std::string$>$ to remember some of the last commands) from which words
can be extracted. However, words can also come from other sources. For example words can be
taken out of a file or out of a Program\_Entity object. \CLAC\ can mix these sources
dynamically. While executing a program out of a file, \CLAC\ can be directed to start taking
words from another file or from a Program\_Entity object. When the second level word source is
exhausted, \CLAC\ will return to the original.

This is implemented by defining an abstract base class Word\_Source. Classes derived from this
class support the virtual method \texttt{get\_word()} that returns the next word from the
Word\_Source. There are two special control words that might be returned: source\_exhausted and
source\_paused. The source\_exhausted word informs \CLAC\ that the Word\_Source has no words
left and that it should be destroyed. The significance of source\_paused is described below.

To handle proper call/return behavior with word sources, \CLAC\ maintains a stack of pointers to
Word\_Source. When it wants the next word, it takes it from the source on the top of the stack.

\begin{verbatim}
stack<Word_Source *> source;
std::string next_word = (*source.top())->get_word();
\end{verbatim}

Note that because \texttt{get\_word()} is virtual, \CLAC\ need not be concerned what type of
word source is active.

If the word received is source\_exhausted, \CLAC\ pops the stack, deleting the word source that
was on top. Due to a virtual destructor in class Word\_Source, this will cause the appropriate
destructor to run to clean up the actual word source type that was active.

At the bottom of the stack is a Command\_Window object. This object never returns
Source\_Exhausted so the stack is never popped beyond that point.

Note that a Command\_Window must be derived from Word\_Source so that a pointer to a
Command\_Window can be put onto the stack of word sources. However, a Command\_Window must also
be derived from Window so that a pointer to a Command\_Window can be put into the window
manager. command windows must support both the virtual methods \texttt{Get\_Image()} and
\texttt{get\_word()}! This is easily done using multiple inheritance.

Similar comments apply to class Program\_Entity. It must be derived from both Entity (so it can reside on the stack, etc) as well as from Word\_Source (so it can be a source of words).

The main loop of \CLAC\ is in the window manager. When the window manager sends a RETURN
keystroke to the command window, the command window calls the word loop function. The word loop
then begins reading words out of the current word source (the command window). Some of those
words may cause the word loop to build new word sources on the stack. These secondary source
would be read until they are exhausted. Eventually the command window returns source\_paused.
This means that the word loop should return; it does not mean that the command window is
exhausted. The word loop returns to the command window which returns to the window manager's
main loop. More keystrokes are collected and the process repeats.

Note that there is nothing stopping the word loop from building another command window on the
stack of word sources. In fact this is a useful facility for letting the user input values
during the execution of a program. Command\_Window's constructors take appropriate parameters so
that the behavior of a command window can be modify appropriately.

\section{\CLAC\ Script}

% -------------
% Chapter Break
% -------------
\chapter{User Interface Considerations}

\end{document}
