
\chapter{\CLAC\ as a Stack Machine}

To get the most out of this tutorial, you should play with \CLAC as you work through this
document. There is nothing quite like trying things out as you go along.

\CLAC\ does all of its calculations (or ``claculations'' as we like to say) on a stack. Think of
a stack of books. The top book is on stack level one. The book right below it is on stack level
two. If you put a new book on the stack, the new book becomes stack level one. The old stack
level one becomes stack level two.

In \CLAC\ the items on the stack are numbers and other objects that can be manipulated by \CLAC,
not books. When you do a calculation, you must first put values on the stack, and then operate
on them.

\CLAC\ displays the items the top several stack levels just before it prints its prompt. The
number of stack levels displayed is configurable \pet{Show how this is done.}, however stack
level one is always displayed at the bottom. This may seem counter-intuitive, but as you will
see it makes it easier to understand how \CLAC\ performs its operations.

For example, when you start \CLAC, you will output such as the following:

\begin{verbatim}
     4: --- : 
     3: --- : 
     2: --- : 
     1: --- : 
    => 
\end{verbatim}    

When you enter the value 1234, it is pushed into stack level one, and the output becomes:

\begin{verbatim}
     4: --- : 
     3: --- : 
     2: --- : 
     1: INT : 1234
    => 
\end{verbatim}    

If you then enter the value 3.14, it is also pushed into stack level one. The previously
entered value becomes stack level two. The output becomes:

\begin{verbatim}
     4: --- : 
     3: --- : 
     2: INT : 1234
     1: FLT : 3.140
    => 
\end{verbatim}

The numbers in the first column show the stack levels. The second column is an abbreviation of
the data type currently stored in that level. Here INT mean integer and FLT means floating point
number. \CLAC\ has twelve data types, each with its own three letter abbreviation. You will meet
most of these types in this tutorial.

You can add the two values in stack levels one and two by entering the ``+'' command at the
prompt. The sum is returned to stack level one where it can be used in future calculations.
After adding the values in the example above, the output becomes:

\begin{verbatim}
     4: --- : 
     3: --- : 
     2: --- : 
     1: FLT : 1237.140
    => 
\end{verbatim}

In this case the INT value was implicitly converted to FLT and the addition was done as the
addition of two floating point values. \CLAC\ defines a number of implicit type conversions
that are designed to not get in your way when you are working with \CLAC. Explicit type 
conversion commands are also provided. You will see examples of them being used later.

This example shows four stack levels of which just two are used. However, \CLAC's stack is as
large as it needs to be until \CLAC\ runs out of memory. During interactive use you will not
likely need more than a few levels to hold temporary results. However, a highly recursive \CLAC\
program might require significantly more stack space during its execution.

\CLAC\ allows you to type multiple ``words'' on the command line separated by spaces. Each word
is processed one at a time, but the stack, and a new prompt, is not displayed until the entire
command line has been processed. For example, consider the following command line:

\begin{verbatim}
    => 7.23 0.41 +
\end{verbatim}

This pushes 7.23 onto the stack, pushes 0.41 onto the stack, and then invokes the ``+''
operation. After completing the command, \CLAC\ should show you 7.640 in stack level one. The
intermediate stack states are not displayed. As you gain experience with \CLAC, you will
probably find yourself typing longer and longer commands at the prompt. However, if you are ever
uncertain about what's going to happen, you can always break a calculation down into small steps
to make sure it's working as you expect.

Notice that the 1237.140 calculated before is still there (now in stack level two). After all,
the 1237.140 might just be the first part of some larger calculation. Type ``+'' at the command
prompt to compute the sum of 1237.140 and 7.640 (which is 1244.780).

You can type ``clear'' at the command prompt to erase everything in the stack.

\CLAC\ regards its input as an unending stream of space delimited words. The return key is
interesting to \CLAC\ only because it signals your desire to evaluate the words given so far.
The final result is not affected in any way by the number of lines you use to enter the input.
For example, the single command below would cause \CLAC\ to do exactly the same operations as
the several commands above did.

\begin{verbatim}
    => 1234 3.14 + 7.23 0.41 + + clear
\end{verbatim}

Of course, the ``clear'' at the end will cause \CLAC\ to erase the result before you have a
chance to look at it! However, this gives you a feeling for how you can combine commands into
one long command line.

\section{Some Words on the \CLAC\ Command Line}

\pet{The behavior in this section is no longer supported since \CLAC\ currently is now a pure
terminal-mode program. However, this behavior might be implemented again in the future.}

If you've been following along with \CLAC, you have probably noticed that \CLAC\ does not erase
your last command on the command line. This allows you to easily repeat your last command. For
example you can double the number in stack level one by doing

\begin{verbatim}
    => 2.0 *
\end{verbatim}

This pushes 2.0 onto the stack thereby shifting whatever number was at stack level one to stack
level two. The ``*'' directs \CLAC\ to pop two items, multiply them, and push the result back
onto the stack.

Since \CLAC\ does not erase your last command, you can just tap the return key over and over and
watch the number in stack level one double each time.

The command line has a number of other useful features. In particular, you can edit the last
command. The left and right arrow keys move the cursor along the command line. You can use the
home key to jump to the start of the command line and the end key to jump to the end of the
command line. Typing a Ctrl+Arrow combination allows you to jump from word to word on the
command line. Finally, the Ins key toggles you between text insert mode and text replace mode.

The editing ability is quite handy for correcting mistakes. Suppose you type

\begin{verbatim}
     => 2.17 15.8 + 7.93 +
\end{verbatim}

Let's say that the answer you get looks funny. ``Oops,'' you say as you look at your last
command (which \CLAC\ preserves for you), ``that 2.17 was supposed to be 21.7.'' All you have to
do is type the home key, arrow over to the number, fix it, and type return. You can type the
return key when the cursor is anywhere in the command line; you do not need to move it to the
end of the line first.

\CLAC\ even remembers the last few command lines. You can use the up and down arrow keys to
scroll through your command line history. This lets you review some of your previous
calculations. Also, you can rerun your previous calculations after possibly adjusting them.

By the way, if you type a normal key (ie not an editing key) immediately after running a
command, \CLAC\ will immediately erase the command line. We hope you haven't been backspacing
over old command lines to type in new ones!

\section{Back to The Stack}

Here are the commands you must use to get \CLAC\ to do the four basic operations.

\begin{tabbing}
\hspace*{3em}\=+\hspace{3em}\=Adds stack levels 1 and 2.\\
\>              -\>            Subtracts stack level 1 from stack level 2.\\
\>              *\>            Multiplies stack levels 1 and 2.\\
\>              /\>            Divides stack level 2 by stack level 1.\\
\end{tabbing}

All of these operations remove the original values from the stack and replace them with the
single result in stack level one. Notice that the order \CLAC\ uses for subtraction and division
resembles the way you might write the operations on paper (if, that is, you draw your stacks the
way we do). For example if want to calculate $1234 - 34$ the stack will look like this just
before you do the operation:

\begin{verbatim}
     4: --- : 
     3: --- : 
     2: INT : 1234
     1: INT : 34
    => 
\end{verbatim}

\CLAC\ does not allow you to group operations with parenthesis. Although this may sound
surprising, the stack-oriented nature of \CLAC\ makes parenthesis totally unnecessary. Consider
the following calculation.

\begin{displaymath}
     \frac{ 2.34 + 5.67 }{ 6.78 - 8.98 }
\end{displaymath}

First, calculate the numerator.

\begin{verbatim}
    => 2.34 5.67 +
\end{verbatim}

Next, calculate the denominator.

\begin{verbatim}
    => 6.78 8.98 -
\end{verbatim}

Notice that the numerator is left on stack level 2. How nice. Compute the final result by doing

\begin{verbatim}
    => /
\end{verbatim}

Of course, you could have done the whole thing using one command line:

\begin{verbatim}
    => 2.34 5.67 + 6.78 8.98 - /
\end{verbatim}

From now on we will calculate all our examples using one \CLAC\ command line. This will save
space in this tutorial. Remember: you can always break a command down into smaller steps. You
can enter each word one at a time.

Here's another example. Suppose you wanted to calculate

\begin{displaymath}
     \frac { 1.64 + 4.52/(3.49 - 6.71) }{ 5.63 / 7.82 }
\end{displaymath}

Simple.

\begin{verbatim}
    => 4.52 3.49 6.71 - / 1.64 + 5.63 7.82 / /
\end{verbatim}

In general, you want to work from the innermost subexpression toward the outermost
subexpression. Calculate numerators first. The stack will save as many temporary values as you
need.

Try this:

\begin{displaymath}
     \frac{1.11 + (2.22 + (3.33 + 4.44)/6.66 ) / 7.77 }{8.88}
\end{displaymath}

Cake.

\begin{verbatim}
    => 3.33 4.44 + 6.66 / 2.22 + 7.77 / 1.11 + 8.88 /
\end{verbatim}

This would be a good time to work through the exercises below. The solutions are at the end of
this tutorial.

\section{Still More Stack Tricks}

Generally you should calculate numerators first. This is because the ``/'' operation expects the
numerator to be in stack level 2. If you calculate the numerator first, that is usually where it
will end up.

Suppose you forget this rule. No problem. \CLAC\ offers a number of words that allow you adjust
the stack. For example ``swap'' exchanges stack levels one and two. This is handy for those
times where you get the numerator and denominator backwards. For example

\begin{displaymath}
     \frac{ 4.45 + 6.77 }{ 9.87 - 2.33 }
\end{displaymath}

This can be calculated using

\begin{verbatim}
    => 9.87 2.33 - 4.45 6.77 + swap /
\end{verbatim}

Suppose you calculate a number into stack level one which is wrong. You want to forget it. The
``drop'' operation pops stack level one and throws it away.

\begin{verbatim}
    => 4.45 6.77 + 9.87 2.23 -
\end{verbatim}

Oops, that 2.23 was supposed to be a 2.33. Let me recalculate the denominator\ldots

\begin{verbatim}
    => drop 9.87 2.33 - /
\end{verbatim}

There are several \CLAC\ commands that just rearrange items on the stack. They are often
necessary in complex calculations to move the appropriate numbers into position. It isn't always
easy to plan a large calculation properly from the start.

The simplest such operation is ``drop'' that we have already discussed. It merely deletes the
item in stack level one and shifts all the other objects down. It is useful to disposing of a
calculated result that you no longer care about.

The ``dup'' command duplicates the item in stack level one so that both stack levels one and two
are identical. All other items are shifted up one level. This command is useful to avoid typing
in a number more than once should it appear more than once in a calculation.

\pet{Talk about roll, rolld, rot}

\section{Scientific Functions}

\CLAC\ supports all the usual scientific functions. As you might expect, each function takes its
argument from stack level one and returns its result to stack level one. For example:

\begin{verbatim}
    => 45.67 sin
\end{verbatim}

computes the sine of 45.67 degrees. This assumes \CLAC\ is in degree mode. We will talk about
the various calculator modes later in this tutorial. Here is a list of the most common functions
that are built into \CLAC:

\begin{tabbing}

\hspace*{3em}\=sin, cos, tan\hspace{5em}\=Trig functions.\\
\>             asin, acos, atan\>         Inverse trig functions.\\
\>             ln, log\>                  Natural log and base 10 log.\\
\>             exp, exp10\>               $e^{x}$ and $10^{x}$.\\
\>             sqrt, sq\>                 Square root and square.\\
\end{tabbing}

For example, suppose you wanted to calculate the distance between the two points $(4.556,
6.778)$ and $(1.224, -7.936)$ in the cartesian plane. The answer is, of course

\begin{verbatim}
    => 4.556 1.224 - sq 6.778 -7.936 - sq + sqrt
\end{verbatim}

The hyperbolic cosine is given by

\begin{displaymath}
     \cosh (x) = \frac{ e^{x} + e^{-x} }{ 2 }
\end{displaymath}

Let's calculate $\cosh(1.34)$

\begin{verbatim}
    => 1.34 dup exp swap neg exp + 2 /
\end{verbatim}

Supposedly the following is an identity

\begin{displaymath}
     \frac { \sin(A) + \sin(B) }{ \sin(A) - \sin(B) } =
       \frac { \tan((A+B)/2) }{ \tan((A-B)/2) }
\end{displaymath}

Shall we see if this is true for $A = 22.3$ degrees and $B = 37.8$ degrees? First let's get the
left hand side:

\begin{verbatim}
    => 22.3 sin dup 37.8 sin dup neg 4 roll + 3 rolld + swap /
\end{verbatim}

We'll be a little less esoteric about the right hand side by resorting to entering the angles
twice (see if you can do it without entering the angles twice).

\begin{verbatim}
    => 22.3 37.8 + 2 / tan 22.3 37.8 - 2 / tan /
\end{verbatim}

Now the value in stack level one and the value in stack level two should be the same which means
their difference should be (close to) zero.

\begin{verbatim}
    => -
\end{verbatim}

When you are done using \CLAC\ you can exit the program using the ``quit'' command:

\begin{verbatim}
    => quit
\end{verbatim}

This introductory chapter should give you enough information to do useful calculations with \CLAC.
The additional chapters of this tutorial will explore many more features of the program, including
some advanced features.

Enjoy!
